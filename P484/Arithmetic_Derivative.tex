\documentclass[11pt]{article}

\usepackage{sectsty}
\usepackage{graphicx}
\usepackage{mathtools}
\usepackage{hyperref}
\usepackage{amssymb}

\numberwithin{equation}{section}

% Margins
\topmargin=-0.45in
\evensidemargin=0in
\oddsidemargin=0in
\textwidth=6.5in
\textheight=9.0in
\headsep=0.25in

\title{ Arithmetic Derivative } 
\author{ Author }
\date{Novemer 29, 2025}

\begin{document}
\maketitle	

% Optional TOC
\tableofcontents
\pagebreak

%--Paper--

\section{ Problem 484 }
Find the following sum:
\begin{equation}
    \sum^{5 \cdot 10^{15}}_{k = 2} \text gcd(k, k') 
\end{equation}
From the bounds of gcd, our sum has to between $N = 5 \cdot 10^{15}$ and $\frac{N(N+1)}{2}$.
%
%
%
%
%
\section{ Arithmetic Derivative }
The arithmetic derivative is defined as follows:
\begin{equation}
\begin{aligned}
    p' &= 1 \text{ for any prime p} \\
    (ab)' &= a'b + ab'
\end{aligned}
\end{equation}



\subsection{ Arithmetic Derivative of 1}
The derivative of one can be found by looking at the base case of $p'$, take the following
$(1 \cdot p)' = 1' \cdot p + p' \cdot 1 = 1$, $(1')$ must be 0.



\subsection{ Arithmetic Derivative of Powers of p}
 
\begin{equation}
    (p^n)' = np^{n-1}    
\end{equation}
To prove this take the base case to be $p'=1$, and assume formula above. Now using proof by
induction, show that $p^{n+1}$ has the same form.
\begin{equation}
\begin{aligned}
    (p^{n+1})' &= (pp^n)' \\
               &= p'p^n + p(p^n)' \\
               &= p^n + npp^{n-1} \\
               &= (n+1)p^{n+1}
\end{aligned}
\end{equation}



\subsection{ Arithmetic Derivative of a Natural Number}

Let $k \in \mathbb{N}$ have a prime factorization $p_1^{\alpha_1} \cdots p_n^{\alpha_n}$ 
where $p_i$ is some prime and $\alpha_i \in \mathbb{N}$ is its corresponding power.
\begin{equation}
    k' = \sum^n_{i=1} \alpha_i \frac{k}{p_i}
\end{equation}
To prove this take the base case $(p^{\alpha_i}_i)' = \alpha_i p^{\alpha_i - 1}_i$, and assume the formula
above. Now using proof by induction, show that $L' = (kp^{\alpha_{n+1}}_{n+1})'$ has the same form.
\begin{equation}
\begin{aligned}
    (kp^{\alpha_{n+1}}_{n+1})' &= (p^{\alpha_{n+1}}_{n+1})'k + p^{\alpha_{n+1}}_{n+1} k' \\
                               &= \alpha_{n+1}p^{\alpha_{n+1} - 1}_{n+1}k + p^{\alpha_{n+1}}_{n+1} \sum^n_{i=1} \alpha_i \frac{k}{p_i} \\
                               &= \alpha_{n+1}\frac{L}{p_{n+1}} + \sum^n_{i=1} \alpha_i \frac{L}{p_i} \\
                               &= \sum^{n+1}_{i=1} \alpha_i \frac{kp^{\alpha_{n+1}}_{n+1}}{p_i} 
\end{aligned}
\end{equation}
It can be seen that $k'$ has a factor of $p_1^{\alpha_1 - 1} \cdots p_n^{\alpha_n - 1}$ in each part of 
its sum.
\begin{equation}
    k' = p_1^{\alpha_1 - 1} \cdots p_n^{\alpha_n - 1} \sum^n_{i=1} \alpha_i \frac{p_1 \cdots p_n}{p_i}
\end{equation}



\subsection{The GCD of k and its Arithmetic Derivative}
Using the closed form formula for $k'$ we can say that the $p_1^{\alpha_1 - 1} \cdots p_n^{\alpha_n - 1} \leq \text gcd(k, k')$.
To remove the inequality, $\sum^n_{i=1} \alpha_i \frac{p_1 \cdots p_n}{p_i}$ needs to be analyzed. If the sum
has a factor of $p_i$ then we can update $p_i^{\alpha_i - 1}$ to  $p_i^{\alpha_i}$ in our $\text gcd(k, k')$ calculation.
So when does $p_i \mid \sum^n_{i=1} \alpha_i \frac{p_1 \cdots p_n}{p_i}$, for all the terms $j \neq i$ there is exactly
one $p_i$ in them. Then when would $p_i | \alpha_i \frac{p_1 \cdots p_n}{p_i}$? There will be at least one factor of $p_i$
when $p_i \mid \alpha_i$. So when $\alpha_i = p_i * j$ $\forall j \in \mathbb{N}$ we get an extra factor of $p_i$ in our $\text gcd(k, k')$.
We will introduce a new function $g(k) = \text{gcd}(k, k')$ to help with readability.
\begin{equation}
    g(k)
    =
    \text{gcd}(k, k') 
    = 
    \prod^n_{i=1} 
    \begin{cases}
			p^{\alpha_i - 1}_i, & \text{if $p_i \nmid \alpha_i$} \\
            p^{\alpha_i}_i,     & \text{if $p_i \mid \alpha_i$} \\
	\end{cases}
\end{equation}
%
%
%
%
%
\section{Properties of g(k)}

The function $g(k)$ is a multiplicative function when partitioned by the $p^{\alpha_i}_i$.
\begin{equation}
    g(p_1^{\alpha_1} \cdots p_n^{\alpha_n}) = g(p_1^{\alpha_1}) \cdots g(p_n^{\alpha_n})
\end{equation}
\newline
The $\sum g(k)$ can be factored if all the arguments share the same factor of $p_i$ and $\alpha_i$
\begin{equation}
\begin{aligned}
    & g(p_1^{\alpha_1} \cdots p_n^{\alpha_n} \cdot q_1^{\beta_1} \cdots q_m^{\beta_m}) 
    + 
    g(p_1^{\alpha_1} \cdots p_n^{\alpha_n} \cdot r_1^{\gamma_1} \cdots r_o^{\gamma_o}) \\
    &= 
    g(p_1^{\alpha_1} \cdots p_n^{\alpha_n})[g(q_1^{\beta_1} \cdots q_m^{\beta_m}) + g(r_1^{\gamma_1} \cdots r_o^{\gamma_o})]
\end{aligned}
\end{equation}
%
%
%
%
%
\section{Creating Group G(p)}
To find $\sum g(k)$ we need $k$'s prime factorization which is expensive. So we can try to build
all the $k$'s using the primes which removes the need for prime factorization.


\subsection{Brute Force}
The brute force method of solving our sum would require calculating the prime factorization of each $k$.
\begin{equation}
    O(\sum g(k)) \propto \sum^N_{k=1} \sqrt{k} \approx \int_{1}^{N} \sqrt{x} dx \approx \frac{2}{3} N^{\frac{3}{2}}
\end{equation}
So we must find a procedure that has a significantly lower time complexity. So we must calculate $g(k)$ in terms 
of some other $g(l)$ which saves the number of computations needed.

\subsection{Motivation}
If we have computed $g(p_1^{\alpha_1} \cdots p_n^{\alpha_n})$ then $g(q^\alpha \cdot p_1^{\alpha_1} \cdots p_n^{\alpha_n})
= g(q^\alpha) g(p_1^{\alpha_1} \cdots p_n^{\alpha_n})$. So create groups $G$ such that we can calculate
all the numbers $k$ that have a factor of $q^\alpha$.



\subsection{Building Group G(p)}

Let $p^{\alpha_n} \mid k \in G(p)$. Now, does $G(p_i) \cap G(p_j) = \emptyset$ when $i \neq j$? 
No. There needs to be one more condition. We will require that $q \nmid k \in G(p)$
where $q < p$.
\begin{equation}
    G(p) = \{k \quad \vert \quad 
    p^\alpha \mid k \land q \nmid k \land 1 < k \leq N \quad
    q < p,  \alpha \in \mathbb{N} \}
\end{equation}
The union of all our groups $G(p)$ is all the values of k we need for $\sum g(k)$.
\begin{equation}
    \sum g(k) = \sum_{p_i} \sum_{k \in G(p_i)} g(k)
\end{equation}
Why is $G(p)$ useful? There are $G(p) = \{p\}$; the p's that have this property 
conform to the following:
\begin{equation}
    G(p) = \{p\} \quad \text{iff} \quad p^2 \nleq N
\end{equation}
The smallest element in $G(p)$ is p, the second smallest is $p^2$ since 
$p^2 < p \cdot p_{+1}$ where $p_{+1}$ is the prime immediately after p. For our N we get $P^* = 70,710,707$.
\begin{equation}
\begin{aligned}
        \sum_{p = P^*}^{\pi(N)} G(p) &= \pi(N) - \pi(P^*) + 1 \\
        &\approx \frac{N}{\log N} - \frac{P^*}{\log P^*} + 1 \\
        &\approx 138,319,418,975,671 - 3,912,265 + 1 \\
        &\approx 138,319,415,063,407
\end{aligned}    
\end{equation}
Since $\pi(\prescript{1}{}{P^*}) \approx 3,912,265$ ...?. However,
there is a $G(p) = \{p, p^2\}$ (show that there is only one p that allows for this).
\begin{equation}
\begin{aligned}
    G(p) &= \{p, p^2\} \quad \text{iff}  \quad p \cdot p_{+1} \nleq N \\
    p &= \prescript{1}{}{P^*} = 70,710,677  \\
    \sum_{k \in G(\prescript{1}{}{P^*})} g(k) &= 1 + p
\end{aligned}
\end{equation}

\subsection{ Categorizing G(p) into 13 Meta-Groups }
A meta group of n is a group of $G(p)$ such that $\mu(G(p)) = n$. The measure $\mu(G)$ measures the most 
amount of primes a value in G has. For example, $\mu(p) = 1$, $\mu(2 \cdot 3) = 2$, $\mu(2^2 \cdot 3 \cdot 5 \cdot 7) = 4$, etc...
So for the $G(2)$ the product of the first 13 primes is less then N, which is also the most amount of primes in a number in $G$, hence
$\mu(G(2)) = 13$. 
\begin{equation}
    \mu(G(p)) = \underset{k \in G(p)}{\mathrm{argmax}} \quad \mu(k)
\end{equation}
For the problem at hand we can precompute when our measure changes; see table below. The constant's
$\prescript{n}{}{P^*}$ convey that all primes below it have a measure of at least n. The can be formalized 
by $n < \mu(G(p))$ for $\forall p < \prescript{n}{}{P^*}$. Combining all the relations one can derive the closed form
solution to $\mu(G(p))$. The difficulty of computing the sum $\sum_{k \in G(p)} g(k)$ is related to the measure $\mu(G(p))$.
\begin{equation}
\begin{aligned}
    \mu(G(p)) &= 13 \quad &\text{iff}& \quad  p = 2 \\
    \mu(G(p)) &= n  \quad &\text{iff}& \quad  \prescript{n-1}{}{P^*} < p \leq \prescript{n}{}{P^*} \\    
    \mu(G(p)) &= 1  \quad &\text{iff}& \quad  \prescript{1}{}{P^*} \leq p
\end{aligned}
\end{equation}


\begin{center}
\begin{tabular}{||c r @{=} l||} 
 \hline
 $\mu$ & \multicolumn{2}{c||}{$\prescript{n}{}{P^*}$}       \\ [1ex] 
 \hline\hline
 13 & $\prescript{13}{}{P^*}$ & 2          \\ [1ex] 
 \hline 
 12 & $\prescript{12}{}{P^*}$ & 5          \\ [1ex] 
 \hline 
 11 & $\prescript{11}{}{P^*}$ & 11         \\ [1ex] 
 \hline
 10 & $\prescript{10}{}{P^*}$ & 19         \\ [1ex] 
 \hline
 9  & $\prescript{9}{}{P^*}$  & 37         \\ [1ex] 
 \hline
 8  & $\prescript{8}{}{P^*}$  & 73         \\ [1ex] 
 \hline
 7  & $\prescript{7}{}{P^*}$  & 157        \\ [1ex] 
 \hline
 6  & $\prescript{6}{}{P^*}$  & 397        \\ [1ex] 
 \hline
 5  & $\prescript{5}{}{P^*}$  & 1,361      \\ [1ex] 
 \hline
 4  & $\prescript{4}{}{P^*}$  & 8,387      \\ [1ex] 
 \hline
 3  & $\prescript{3}{}{P^*}$  & 170,957    \\ [1ex] 
 \hline
 2  & $\prescript{2}{}{P^*}$  & 70,710,649 \\ [1ex] 
 \hline
 1  & $\prescript{1}{}{P^*}$  & 70,710,677 \\ [1ex] 
 \hline
\end{tabular}
\end{center}

















\pagebreak
\section{}



\end{document}
