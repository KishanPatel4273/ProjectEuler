\documentclass[11pt]{article}

\usepackage{sectsty}
\usepackage{graphicx}
\usepackage{mathtools}
\usepackage{hyperref}
\usepackage{amssymb}

% copied from https://www.overleaf.com/learn/latex/Code_listing
\usepackage{listings}
\renewcommand{\lstlistingname}{Code }

\usepackage{xcolor}

% New colors defined below
\definecolor{codegreen}{rgb}{0,0.6,0}
\definecolor{codegray}{rgb}{0.5,0.5,0.5}
\definecolor{codepurple}{rgb}{0.58,0,0.82}
\definecolor{backcolour}{rgb}{0.95,0.95,0.92}
\lstdefinestyle{mystyle}{
  backgroundcolor=\color{backcolour}, 
  commentstyle=\color{codegreen},
  keywordstyle=\color{magenta},
  numberstyle=\tiny\color{codegray},
  stringstyle=\color{codepurple},
  basicstyle=\ttfamily\footnotesize,
  breakatwhitespace=false,         
  breaklines=true,                 
  captionpos=b,              
  keepspaces=true,                 
  numbers=left,                    
  numbersep=5pt,                  
  showspaces=false,                
  showstringspaces=false,
  showtabs=false,                  
  tabsize=2
}
\lstset{style=mystyle}

% Margins
\topmargin=-0.45in
\evensidemargin=0in
\oddsidemargin=0in
\textwidth=6.5in
\textheight=9.0in
\headsep=0.25in

\title{ Factorial Trailing Digits}
\author{ Author }
\date{Novemer 26, 2025}

\begin{document}
\maketitle	

% Optional TOC
\tableofcontents
\pagebreak

%--Paper--

\section{Formulating Problem}
Problem: Find the last five digits before the trailing zeroes in $N!$
\newline 
\newline
Let $(N_n, b_d)$ where $N_n \coloneq 10^n$, $b_d \coloneq 10^d$, $b_d \leq N_n$ and $n,d \in \mathbb{N}^+$
be the problem of finding the last $d$ digits of before the trailing zeros in $N!$. Then the 
problem we are interested in is $(N_{12}, b_5)$.
\newline 
\newline
Let $z(n)$ be a function that gives the number of trailing zeros in $n$. 
The $z(n!)$ is given by \href{https://en.wikipedia.org/wiki/Legendre%27s_formula}{De Polignac's formula}:
\begin{equation}
    z(n!) = \sum^{ \lfloor log_5(n) \rfloor }_{i=1} \lfloor \frac{n}{5^i} \rfloor 
\end{equation}
\newline 
\newline
Using De Polignac's formula we can remove the trailing zeros to get the digits of interest.
\begin{equation}
    (N_n, b_d) = \frac{N!}{2^{z(N!)} 5^{z(N!)}} \mod b
\end{equation}





\pagebreak
\section{Grid Construction and Introduction to Series}
For the problem $(N_n, b_d)$ define $\lambda \coloneq \frac{N}{b} = 10^{n-d}$. 
Then there is a grid $G$ that can be constructed with dimensions is $b \times \lambda$.
\begin{equation}
\begin{matrix}
    1 & 2 & \cdots & 5 & \cdots & 10 & \cdots & b \\
    b + 1 & b + 2 & \cdots & b + 5 & \cdots & b + 10 & \cdots & b + b \\
    2b + 1 & 2b + 2 & \cdots & 2b + 5 & \cdots & 2b + 10 & \cdots & 2b + b \\
    \vdots & \vdots & \ddots & \vdots & \ddots & \vdots & \ddots & \vdots \\ 
    (\lambda - 1)b + 1 & (\lambda - 1)b + 2 & \cdots & (\lambda - 1)b + 5 & \cdots & (\lambda - 1)b + 10 & \cdots & (\lambda - 1)b + b \\
\end{matrix}
\end{equation}
The product of all the elements of $G$ is $N!$. Let $c_v$ be the $v$th column of $G$. 
Then we define $c_v(i)$ as the $i$th element in column $c_v$. \\
\begin{equation}
    c_v(i) = (i-1)b + v \quad i \in \{1 \cdots \lambda\} \\
\end{equation}
The series $c_v$ is an arithmetic sequence which can be seen by the following, $a_1 = v$ and $a_i = a_{i-1} + b$.  
Then let $p_v$ be the product of the terms in $c_v$ series.
\begin{equation}
    p_v = \prod^\lambda_{i=1} (i-1)b + v
\end{equation}
Then we have the series $p_1, p_2, \cdots p_b$ whose product is $N!$.





\subsection{Building Solution from Column's of G}

Given the series $p_v$ from grid $G$ built for problem $(N_n, b_d)$ 
we can apply sequence lemma LS4 for $v|5$ else apply LS3. This will give
the $\frac{N!}{5^{z(N!)}} \mod b$. Then to get the solution $(N_n, b_d)$ 
we need to use sequence lemma LS3 to remove 2's for each 5 we removed. 
From LS3 we can remove $\lambda$ 2's at a time. However, then we would need
to multiple back the extra 2's that were taken off. This is fine and can 
be done mod b using lemma L1.
\begin{align}
    Z &= \lceil \frac{z(N!)}{\lambda} \rceil \\
    R &= Z - z(N!)
\end{align}
There are $4\frac{b}{10}$ $p_v$ that can have LS4 applied to them. The following construction
hold if $Z \leq 4\frac{b}{10}$ else a construction of a similar form will need to be made which is 
fairly trivial with the machinery built up in the lemmas. The construction is as follows:
\begin{align}    
    S_5 &= \{5, 10, \cdots, b\} \\
    S_2 &\subset \{2, 4, \cdots , b\} \setminus S_5 \\
    |S_2| &= Z \\
    S &= \{1, 2, \cdots , b\} \ \setminus S_2 \cup S_5
\end{align}
Let the $P_n(a_i) \mod b$ be the product of the first n terms of the series $a_i$ mod b (defined by LS2).
Let the $P^5_n(a_i) \mod b$ be the product of the first n terms of the series $a_i$ dropping all 5's mod b (defined by LS3).
\newline
\newline
The construction takes out $\lambda Z$ 2's which is $R$ more than the number of 2's need to be taken
out which is $z(N!)$. So using lemma L1 we can add back $2^R$.
\begin{equation}
    (N_n, b_d) 
    = 
    \prod_{\forall v_i \in S, v_j \in S_2, v_k \in S_5}
    [P_\lambda(c_{v_i}) \mod b] 
    \cdot 
    [P_\lambda(\frac{1}{2}c_{v_j}) \mod b] 
    \cdot 
    [P^5_\lambda(c_k) \mod b] \cdot 2^R 
    \pmod b 
\end{equation}

\vspace*{\fill}
\lstinputlisting[language=Python, firstline=194, lastline=225]{Factorial_Trailing_Digits_P160.py}










\pagebreak
\section{Sequence}

\subsection{Lemma L1}

Let $a,b \in \mathbb{N}$ then we can write $a$, $b$ and $ab$ with their 2's and 5's factored out:
\newline
\newline
$a = \alpha \cdot 2^{\alpha_2} \cdot  5^{\alpha_5}$
\newline
$b = \beta \cdot 2^{\beta_2} \cdot  5^{\beta_5}$ 
\newline
$ab = \lambda \cdot 2^{\gamma} \cdot  5^{\gamma}$ where $\gamma \coloneq \min (\alpha_2 + \beta_2, \alpha_5 + \beta_5)$
\newline

The following two relations hold for any $m$.
\begin{equation}
    \alpha\beta \cdot 2^{\alpha_2 + \beta_2 - \gamma} \cdot  5^{\alpha_5 + \beta_5 - \gamma} 
    \equiv
    \lambda
    \pmod m
\end{equation}

This then allow for the following:
\begin{equation}
    (\frac{a}{2^\gamma 5^\gamma} \mod m) (\frac{b}{2^\gamma 5^\gamma} \mod m)
    \equiv
    \lambda
    \pmod m
\end{equation}





\subsection{Lemma LS1; Sequence is Purely Periodic}

Given a sequence $a_i = (i-1) \cdot r + v$, $a_i \mod m$ where $i,r,v,m \in \mathbb{N}$ 
is purely periodic with period $\tau = \frac{m}{gcd(r, m)}$. First redefine $a_i$ as $a_0 = v$ and $a_i = a_{i-1} + r$.
Then we have $a_{k+l} = a_k + l r$ from the arithmetic progression.
\begin{equation}
    a_{k+l} \equiv a_k \pmod m  \leftrightarrow lr \equiv 0 \pmod m \leftrightarrow m|lr
\end{equation}
Let the smallest $m$ that satisfies $m|lr$ be $x$. Then $m|xr$ which implies that $m = \text{gcd}(m, xr)$.
One possibility of $x = m$, is there smaller number that satisfies the condition?
If $xr$ have a common divisor $d = \text gcd(x, r)$ then we get $x^* r^* d^2$ sill divides $m$ by definition.
Since $m|d$ hold always we can pull $d$ out of out $x = m$ we can reduce $x$ to $\frac{m}{\text gcd(r, m)}$ which 
is less than or equal to $m$ and since there is no other larger number that we can pull out of both $x^*$ and $r^*$,
$x$ is the smallest it can be then i.e. it is our period $\tau$.
[\href{https://math.stackexchange.com/questions/1861199/progressions-modulo-n}{1}]





\pagebreak
\subsection{Lemma LS2; Product of Periodic Sequence}
Given a periodic sequence $a_i$ with period $\tau$ and the product of the first $n$ terms be $p_n$,
then following allows for the computation of higher large $n$.
\begin{equation}
    \prod^n_{i=1} a_i = p_\tau^{\lfloor \frac{\lambda}{\tau} \rfloor} \cdot p_{n \text{ mod $\tau$}}
\end{equation}

\vspace*{\fill}
\lstinputlisting[language=Python, firstline=123, lastline=141]{Factorial_Trailing_Digits_P160.py}






\pagebreak
\subsubsection{Series: (i-1)b + v mod b}
We have the sequence $c_v(i) = (i-1)10^d + v = (i-1)2^d 5^d + v$. 
We will use lemma L1, which requires $\text gcd(c_v, 5) = 1$. The only way 
$\text gcd(c_v, 5) > 1$ is if $\text gcd(v, 5) > 1$. So for the following we
will deal with cases in which $\text gcd(v, 5) = 1$ that is to say $v$ cannot be a multiple of 5. 
Since all our elements in our sequence not divisible by 5 lemma L1 can be used to get $\mod b$ product. 
Since our sequence is arithmetic we can use sequence lemma LS1 to get our 
sequence  $\mod b$ has period $1 = \frac{b}{\text gcd(b,b)}$ and $c_v \mod b = v$.
Using sequence lemma LS2 since we have a periodic sequence we get the following:

\begin{align}
    \prod^n_{i=1} a_i &\equiv (p_\tau \mod b)^{\lfloor \frac{\lambda}{\tau} \rfloor} \cdot (p_{n \text{ mod $\tau$}} \mod b) \pmod b \\
    p_v &\equiv v^\lambda \pmod b
\end{align}




\subsubsection{Series: [(i-1)b/2 + v]/2 mod b}
For the sequence $c_v(i) = (i-1)10^d + v = (i-1)2^d 5^d + v$ we want the mod $b$ 
of the product of $\frac{1}{2} c_v(i)$  first $\lambda$ terms. Since $v|2$ and 
$\text gcd(v, 5) = 1$ we can use L1, LS1 and LS2. 
\begin{equation}
    \frac{1}{2} c_v(i) = (i-1)2^{d-1} 5^d + \frac{v}{2}
\end{equation}
Using LS1 we get the period of the new sequence as $\frac{2^d 5^d}{\text gcd(2^{d-1} 5^d, 2^d 5^d)} = 2$.
Then using LS2 we get the following and L1.
\begin{equation}
% \begin{align}
    p_v 
    \equiv 
    [\frac{v}{2} \cdot (2^{d-1} 5^d + \frac{v}{2}) \mod b]^{ \lfloor \frac{\lambda}{2} \rfloor} \cdot (\frac{v}{2})^{\lambda \mod b}
    \pmod b
% \end{align}
\end{equation}





\pagebreak
\subsection{Lemma LS3: Product of Series dropping 5's mod b}
Let the series be of the form $a_i = (i-1) \cdot r + v$. From sequence lemma LS1 we know that
$a_i$ is purely periodic with period $\tau = \frac{b}{\text gcd(r, b)}$. Find the following product:
\begin{equation}
    \prod^\lambda_{i=1} \frac{a_i}{5^{p^i_5}} \pmod b
\end{equation}
where $p^i_5$ is the exponent for the factor 5 for $a_i$'s factorization.
Let $r = r^* 5^{p_r}$ and $v = v* 5^{p_v}$ where $\text gcd(r^*, 5) =  gcd(v^*, 5) = 1$. Then
the series $a^*_i = (i-1) \cdot r^* + v^*$ product of first $\lambda$ terms dropping 5's mod b is
the same as our original problem.
\newline
\newline
case: $p_v < p_r$
\newline
\newline
The series will not have any multiple of 5's i.e. $\text gcd(a^*_i, 5) = 1 \text{} \forall i \in \mathbb{N}$.
Therefore sequence lemma LS2 can be used to find the product mod b.
\newline
\newline
case: $p_r \leq p_v $
\newline
\newline
The series will of the following form $a^*_i = (i-1) \cdot r^* + v^* 5^{p_v - p_r}$. When does $a^*_i|5$?
The second term will have end in $\{5\}$ when $p_r < p_v$ and will be $\{1,2,3,4,6,7,8,9\}$ when $p_r = p_v$, 
it could end in 0 iff $v = 0$. Without specifics analysis to find to smallest i for $a^*_i|5$ must be done manually. 
To do this take the array $[0, \cdots, 9] (r^* \mod 10) + (v^* \mod 10)$ and find the first instance of
a multiple of 5, the index (1-indexed) of this position be $O_5$. Then $a^*_{O_5} | 5$ is the first element in the series
that is divisible by 5. The sequence $a^*_{O_5}, a^*_{O_5 + 5}, \cdots$ are all the terms that are multiple of 5. 
\newline
\newline
Therefore, we can use sequence lemma LS2 for the terms in $a^*_i$ where $i \notin \{O_5, O_5 + 5, \cdots\}$ iff $O_5 \leq 5 \land \tau|5$ .
The condition on the period of the sequence is from the fact that each period of the sequence needs to take out each instance
of $a^*_i|5$ and need to line up with with in the period; this is true if the period of $a^*_i|5$ 5 is a factor of $\tau$ and 
$i \neq O_5 + 5k \text{   } \forall k \in \mathbb{Z}$. 
\newline
\newline
Therefore here does not seem to be a pattern for the terms $a^*_{O_5 + 5k}$ this can be seen by $a^*_{O_5 + 5k}|5^2$ is periodic
but it will have its own $O_5^2$ that needs to be calculated. There will be at most $n_O = \lfloor \log_5 \lambda \rfloor$ offsets 
that will needed to be calculated. And each offset $O_{5^{p_O}}$ the following must
be true:
\begin{align}
    &\tau|5^{p_O} \\
    &O_{5^{p_O}} \leq \tau \\
    &O_{5^{p_O}} \leq 5^{p_O} \\
    &5^{p_O} \leq \tau \\
\end{align}
These condition are not generally going to be hold for $1 < p_O$, so
the series $a^*_{ O_5 + 5k}$ will need to be brute forced.

\vspace*{\fill}
\lstinputlisting[language=Python, firstline=143, lastline=191]{Factorial_Trailing_Digits_P160.py}





\pagebreak
\section{Helper Methods}
\lstinputlisting[language=Python, firstline=1, lastline=121]{Factorial_Trailing_Digits_P160.py}



\end{document}